\documentclass[10pt, a4paper,english,spanish,hidelinks]{article}
\usepackage{amsmath}
\usepackage{amsfonts}
\usepackage{amssymb}
\usepackage{caratula}
\usepackage[spanish, activeacute]{babel}
\usepackage[usenames,dvipsnames]{color}
\usepackage[width=15.5cm, left=3cm, top=2.5cm, height= 24.5cm]{geometry}
\usepackage{graphicx}
\usepackage[utf8]{inputenc}
\usepackage{listings}
\usepackage{multicol}
\usepackage{subfig}
\usepackage{float}
\usepackage{color,hyperref}


\usepackage{listings}
\usepackage{babel}
\usepackage{url}
\usepackage{lscape}
\parindent = 15 pt
\parskip = 11 pt

\usepackage{fancyhdr}
\usepackage{hyperref}
\usepackage{amsmath}
\usepackage{amsfonts}
\usepackage{amssymb}
\usepackage[utf8]{inputenc}
\usepackage{graphicx}
\usepackage{caption}
\usepackage{color}
\usepackage{appendix}
\usepackage{fancyhdr}


\materia{Paradigmas de Lenguajes de Programación}

\titulo{Trabajo Práctico 1}
\fecha{09 de Septiembre de 2014}
\grupo{Grupo tas lokito}
\integrante{Gauder María Lara}{027/10}{marialaraa@gmail.com}
\integrante{Mataloni Alejandro}{706/07}{amataloni@gmail.com}
\integrante{Reartes Marisol}{422/10}{marisol.r5@hotmail.com}


\begin{document}
\pagestyle{myheadings}
\maketitle
\markboth{Paradigmas de Lenguajes de Programación}{}

\thispagestyle{empty}
\tableofcontents
\newpage
\section{Código y comentarios}
A continuación se muestra cada función con su explicación.

\subsection{Ejercicio 1}
La función belongs define si cierto elemento pertenece a las claves del diccionario dado como argumento.

\begin{verbatim}
belongs :: Eq k => k -> Dict k v -> Bool
belongs k d = not (null [x | x <- d, fst x == k])

(?) :: Eq k => Dict k v -> k -> Bool
d ? k = belongs k d
\end{verbatim}

Ejemplo:

\begin{verbatim}
[("Pedro", ["Berlin", "Paris"]), ("Juan", ["Peru", "Amsterdam"])] ? "Juana" 
--> False
\end{verbatim}

\subsection{Ejercicio 2}
La función get retorna el significado correspondiente a la clave dada como argumento, se asume que el mismo se encuentra definido.
\begin{verbatim}
get :: Eq k => k -> Dict k v -> v
get k d = head ([snd x | x <- d, fst x == k])

(!) :: Eq k => Dict k v -> k -> v 
d ! k = get k d
\end{verbatim}
Ejemplo:
\begin{verbatim}
get "Argentina" [("Chile", 5), ("Argentina", 9), ("Uruguay", 3)]
--> 9

[( " calle " ," San Blas " ) ,( " ciudad " ," Hurlingham " )] ! " ciudad "
--> " Hurlingham "
\end{verbatim}

\subsection{Ejercicio 3}
La función insertWith incorpora un nuevo significado a una clave dada, utilizando como función de incorporación a aquella dada como argumento. Si no existe la clave en el diccionario, entonces se la misma será agregada.
\begin{verbatim}
insertWith :: Eq k => (v -> v -> v) -> k -> v -> Dict k v -> Dict k v
insertWith fadd k v d
  | not (d ? k) = d ++ [(k,v)]
  | d ? k       = [if (fst x == k) then (fst x, fadd (snd x) v ) else x | x <- d]
\end{verbatim}
Ejemplo:

\begin{verbatim}
insertWith (++) 1 [99,188] [(1,[1]),(2,[2])]
--> [(1,[1,99,188]),(2,[2])]
\end{verbatim}

\subsection{Ejercicio 4}
En este caso, se ordenará los datos del array pasado como parámetro para que pase a tener una estructura de tipo Diccionario. Se asociará cada clave con la lista de todos los valores que tenía en el array original.
\begin{verbatim}
groupByKey :: Eq k => [(k,v)] -> Dict k [v]
groupByKey l = foldr (\x xs -> insertWith (++) (fst x) [(snd x)] xs) [] l
\end{verbatim}
Ejemplo:
\begin{verbatim}
groupByKey [("fruta", "Banana"), ("carne", "Picada"), ("verdura", "Espinaca")]
--> [("fruta", ["Banana"]), ("carne", ["Picada"]),("verdura", ["Espinaca"])]
\end{verbatim}




\subsection{Ejercicio 5}
En unionWith se espera que a partir de dos diccionarios y una función de unificación, se combinen ambos para obtener un único diccionario resultante.
\begin{verbatim}
unionWith :: Eq k => (v -> v -> v) -> Dict k v -> Dict k v -> Dict k v
unionWith f dict = foldr (\x rec_dict -> insertWith f (fst x) (snd x) rec_dict) dict
\end{verbatim}
Ejemplo:
\begin{verbatim}
unionWith (+) [("Algodon", 20), ("Azucar", 10)]  [("Miel", 5),("Algodon", 5)]
--> [("Algodon", 25), ("Azucar", 10),("Miel", 5)]
\end{verbatim}


\subsection{Ejercicio 6}
Tendrá como objetivo que a partir de una lista divide la carga de la misma de manera balanceada entre una determinada cantidad de sublistas. 

\begin{verbatim}
distributionProcess :: Int ->  [a] -> [[a]]
distributionProcess n = foldr (\x rec -> (tail rec) ++ [x : (head rec)]) (replicate n [])
\end{verbatim}

Ejemplo:
\begin{verbatim}
distributionProcess 1 [1, 4, 100, 104]
--> [[1, 4, 100, 104]]
\end{verbatim}

\subsection{Ejercicio 7}
La función mapperProcess deberá aplicar el tipo Mapper a cada uno de los elementos pasados en el array por parámetro y luego agrupar los resultados por claves. 
\begin{verbatim}
mapperProcess :: Eq k => Mapper a k v -> [a] -> [(k,[v])]
mapperProcess m la = groupByKey (concat [m e | e <- la])
\end{verbatim}

Ejemplo:
\begin{verbatim}
mapperProcess (\x -> [(x,1)]) ["Pablo", "Pedro", "Pepe", "Pedro", "Pedro", "Guillermo", "Aldana"] 
--> [("Pablo",[1]), ("Pedro",[1,1,1]), ("Pepe",[1]),("Guillermo",[1]), ("Aldana",[1])]
\end{verbatim}
     
\subsection{Ejercicio 8}
combinerProcess consiste en unificar los resultados de cada sublista pasada por parámetro agrupándolos por clave y ordenandolos de manera creciente según la clave. 
\begin{verbatim}
combinerProcess :: (Eq k, Ord k) => [[(k, [v])]] -> [(k,[v])]
combinerProcess l = sortBy (comparing $ fst) (foldr (\x rec -> unionWith (++)  x rec) [] l)
\end{verbatim}

Ejemplo:
\begin{verbatim}
combinerProcess [[("Pablo",[1]), ("Pedro",[1,1,1])], [("Pepe",[1])],[("Guillermo",[1]), 
("Aldana",[1])]]
--> [("Aldana",[1]), ("Guillermo",[1]),("Pablo",[1]), ("Pedro",[1,1,1]), ("Pepe",[1])]
\end{verbatim}

      
\subsection{Ejercicio 9}
La función reducerProcess aplica el tipo Reducer a cada para de elementos de la lista pasada como argumento y luego combina los resultados para obtener una única lista. 
\begin{verbatim}
reducerProcess :: Reducer k v b -> [(k, [v])] -> [b]
reducerProcess r l = concat [r e | e <- l]
\end{verbatim}

Ejemplo:
\begin{verbatim}
reducerProcess (\(x,y) -> [sum y])  [("Alvear",[3,5,1,32]),("Cabildo",[4,6,4,4,4]),
("Independencia",[1,1])] 
--> [41, 22, 2]
\end{verbatim}
      

\subsection{Ejercicio 10}
La función mapReduce combina todas las instrucciones desarrolladas de la Sección 2, para poder implementar una técnica de MapReduce, que permite en tiempo óptimo procesar grandes cantidades de datos.
\begin{verbatim}
mapReduce :: (Eq k, Ord k) => Mapper a k v -> Reducer k v b -> [a] -> [b]
mapReduce m r l = reducerProcess r (combinerProcess [mapperProcess m x | x <- dist])
  where dist = distributionProcess 100 l
\end{verbatim}

Ejemplo:
\begin{verbatim}
mapReduce (\(nom,ciudad) -> [(ciudad,1)]) (\(ciudad,cant) -> [(ciudad, sum cant)])
[("Pablo","Berlin"), ("Gabriela","Amsterdam"), ("Taihú","Cairo"), ("Pablo", "Cairo"),("Taihú","Amsterdam"), ("Juan","Amsterdam")] 
--> [("Amsterdam", 3), ("Berlin", 1), ("Cairo", 2)]
 \end{verbatim}

\subsection{Ejercicio 11}
La función visitasPorMonumento deberá, a partir de un array de nombres de monumentos repetidos generar un Diccionario que indique la cantidad de repeticiones que hay por cada uno. 
\begin{verbatim}
visitasPorMonumento :: [String] -> Dict String Int
visitasPorMonumento m = mapReduce (\l -> [(l,1)]) (\(x,y) -> [(x,sum y)]) m
\end{verbatim}

Ejemplo:
\begin{verbatim}
 visitasPorMonumento [ "Obelisco", "Cid", "San Martín", "Cid", "Bagdad Bridge", "Cid", 
 "San Martín", "Obelisco"] 
--> [("Obelisco",2), ("Cid",3), ("San Martín",2), ("Bagdad Bridge", 1)]
\end{verbatim}
 

\subsection{Ejercicio 12}
En este caso, monumentosTop deberá, a partir de una lista de nombres de monumentos repetidos según la cantidad de visitas, ordenarlos por cantidad de visitas, de mayor a menor, sin repetidos. 
\begin{verbatim}
monumentosTop :: [String] -> [String]
monumentosTop m = mapReduce (\(x,y) -> [(-y,x)]) (\(x,y) ->  y) (visitasPorMonumento m)
\end{verbatim}

Ejemplo:
\begin{verbatim}
monumentosTop [ "Obelisco", "Obelisco", "Cid", "San Martin", "San Martin", "San Martin", 
"Obelisco", "Obelisco"]
--> ["Obelisco", "San Martin", "Cid"]
\end{verbatim}


\subsection{Ejercicio 13}
Indica la cantidad de monumentos existentes para cada país que tenga.
\begin{verbatim}
monumentosPorPais :: [(Structure, Dict String String)] -> [(String, Int)]
monumentosPorPais s = mapReduce (\(s,d) -> 
[(d ! "country", initNumber s)]) (\(x,y) -> if sum y /= 0 then [(x,sum y)] else []) s

initNumber :: Structure -> Int
initNumber Monument = 1
initNumber Street = 0
initNumber City = 0
\end{verbatim}

Ejemplo:
\begin{verbatim}

monumentosPorPais [ (Monument, [("Antiguedad","muchos años"),("name","Catedral de Leon"),
				("country", "España")]), \\
	  (Monument, [("name","Cristo-Rei de Almada"),("Renovacion", "ninguna"),("country", "Portugal")]), \\
	  (Monument, [("name", "Monasterio de Piedra"),("country", "España")]), \\
	  (Monument, [("name", "Iglesia de San Francisco"),("country", "Argentina")]), \\
	  (Street, [("name", "Corrientes"),("country", "Argentina")]), \\
	  (City, [("name", "Berlina"),("country", "Alemania"),("habitantes", "Dos millones")]), \\
	  (Street, [("name", "Armenia"),("country", "Argentina"),("Barrio", "Palermo")])] \\
--> [("Argentina",1),("España",2),("Portugal",1)]

\end{verbatim}
\end{document}
