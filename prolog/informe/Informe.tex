\documentclass[10pt, a4paper,english,spanish,hidelinks]{article}
\usepackage{amsmath}
\usepackage{amsfonts}
\usepackage{amssymb}
\usepackage{caratula}
\usepackage[spanish, activeacute]{babel}
\usepackage[usenames,dvipsnames]{color}
\usepackage[width=15.5cm, left=3cm, top=2.5cm, height= 24.5cm]{geometry}
\usepackage{graphicx}
\usepackage[utf8]{inputenc}
\usepackage{listings}
\usepackage{multicol}
\usepackage{subfig}
\usepackage{float}
\usepackage{color,hyperref}


\usepackage{listings}
\usepackage{babel}
\usepackage{url}
\usepackage{lscape}
\parindent = 15 pt
\parskip = 11 pt

\usepackage{fancyhdr}
\usepackage{hyperref}
\usepackage{amsmath}
\usepackage{amsfonts}
\usepackage{amssymb}
\usepackage[utf8]{inputenc}
\usepackage{graphicx}
\usepackage{caption}
\usepackage{color}
\usepackage{appendix}
\usepackage{fancyhdr}


\materia{Paradigmas de Lenguajes de Programación}

\titulo{Trabajo Práctico 2}
\fecha{04 de Noviembre de 2014}
\grupo{Grupo tas lokito}
\integrante{Gauder María Lara}{027/10}{marialaraa@gmail.com}
\integrante{Mataloni Alejandro}{706/07}{amataloni@gmail.com}
\integrante{Reartes Marisol}{422/10}{marisol.r5@hotmail.com}


\begin{document}
\pagestyle{myheadings}
\maketitle
\markboth{Paradigmas de Lenguajes de Programación}{}

\thispagestyle{empty}
\tableofcontents
\newpage
\section{Código y comentarios}
A continuación se muestra cada función con su explicación.

\subsection{Funciones auxiliares}

Auxiliar dada en clase
\begin{verbatim}
%desde(+X, -Y).
desde(X, X).
desde(X, Y):-desde(X, Z),  Y is Z + 1.
\end{verbatim}


Se eliminan los elementos duplicados de la primer lista L1.
\begin{verbatim}
%sacarDup(+L1, -L2).
sacarDup([],[]).
sacarDup([X|L],L2) :- member(X,L), !, sacarDup(L,L2).
sacarDup([X|L],[X|L2]) :- not(member(X,L)), sacarDup(L,L2).
\end{verbatim}

Si ambas están definidas, retorna true si las dos son iguales.
Si una está instanciada y la otra no, genera la misma lista.
Y si las dos no están definidas, genera dos listas iguales.

\begin{verbatim}
%sameList(?L1, ?L2).
sameList([],[]).
sameList([X|Xs],[X|Ys]) :- sameList(Xs,Ys).
\end{verbatim}

\subsection{Ejercicio1}

\begin{verbatim}
%esDeterministico(+Automata)
esDeterministico(Automata) :- not(noEsDeterministico(Automata)).
\end{verbatim}

\begin{verbatim}
% noEsDeterministico(+Automata)
noEsDeterministico(Automata) :- transicionesDe(Automata,T),
								  member((O1,E1,D1), T),
								  member((O2,E2,D2),T),
								  O1 = O2, E1 = E2, D1 \= D2.


\end{verbatim}
Ejemplo:

\subsection{Ejercicio 2}

\begin{verbatim}
% estados(+Automata, ?Estados)
estados(Automata, Estados) :- var(Estados), inicialDe(Automata, Si), 
				transicionesDe(Automata, Ts), origenesydstAutomata(Ts, ODs),
				finalesDe(Automata, Fs), append([Si|ODs], Fs, Es), sort(Es, Estados).

estados(Automata, Estados) :- nonvar(Estados), inicialDe(Automata, Si), 
				transicionesDe(Automata, Ts), origenesydstAutomata(Ts, ODs),
				finalesDe(Automata, Fs), append([Si|ODs], Fs, Es),
				forall(member(E, Es), member(E, Estados)).
\end{verbatim}


A partir de las transiciones de un automata, retorna una lista con todos los estados
que participan de alguna transición.

\begin{verbatim}
% origenesydstAutomata(+Ts, ?L).
origenesydstAutomata([], []).
origenesydstAutomata([(O,_,D)|Ts], [O,D|ODs]) :- origenesydstAutomata(Ts, ODs).
\end{verbatim}

Ejemplo:

\subsection{Ejercicio 3}
\begin{verbatim}
% esCamino(+Automata, ?EstadoInicial, ?EstadoFinal, +Camino)
esCamino(A, Si, Si, [Si]) :- estados(A, Estados), member(Si, Estados).
esCamino(A, Si, Sf, [Si|C]) :- last(C,Sf2), Sf = Sf2,
								transicionesDe(A,T), estanEn([Si|C], T).


%estanEn(-L1, +T)
estanEn([O,D], L2) :- member((O,_,D), L2), !.
estanEn([O,D|L1], L2) :- member((O,_,D), L2), estanEn([D|L1], L2). 									
\end{verbatim}

Ejemplo:

\subsection{Ejercicio 4}

No, no es reversible. 
Esto se debe a la forma de recorrer el camino para la verificación. No se tiene
en cuenta el caso de generación del camino.
El predicado genera los posibles caminos cuando no hay ciclo y luego se cuelga.
Cuando hay ciclo, genera algunos caminos pero no todos, ya que se queda en el loop,
y sigue infinitamente generando por la misma rama.


\subsection{Ejercicio 5}

\begin{verbatim}
% caminoDeLongitud(+Automata, +N, -Camino, -Etiquetas, ?S1, ?S2)
caminoDeLongitud(A, 1, [Si], [], Si, Si) :- !, estados(A, Es), member(Si, Es). 			
caminoDeLongitud(A, N, C, Tags, Si, Sf):- N > 1, estados(A, Es), member(Si, Es), member(Sf, Es), transicionesDe(A,T), 
											Nm1 is N-1, crearCamino(Si, Sf, T, C, Nm1, Tags).

\end{verbatim}


Dado dos estados que serán el principio y el final del camino, genera los caminos de
tamaño N con sus respectivas etiquetas. 

\begin{verbatim}
%crearCamino(?Si, ?Sf, +T, -Camino, +N, -Etiquetas).
crearCamino(X, Sf, T, [X,Sf], 1, [E]) :- !, member((X,E,Sf),T).
crearCamino(X, Sf, T, [X|C], N, [E|Etiquetas]) :- Nm1 is N-1, member((X,E,Y),T), crearCamino(Y, Sf, T, C, Nm1, Etiquetas). 
\end{verbatim}

Ejemplo:

\subsection{Ejercicio 6}
Se utiliza Generate & Test.
\begin{verbatim}
% alcanzable(+Automata, +Estado)
alcanzable(A, E) :- inicialDe(A, Si), 
					estados(A,Estados),	length(Estados, N), between(2, N, X),
					caminoDeLongitud(A, X, _, _, Si, E), !.

\end{verbatim}

Ejemplo:


\subsection{Ejercicio 7}
\begin{verbatim}
% automataValido(+Automata)
automataValido(A) :- estados(A, Estados), finalesDe(A, Finales), subtract(Estados, Finales, EstadosNoFinales), 
						inicialDe(A, Inicial), subtract(Estados, [Inicial], EstadosNoIniciales), transicionesDe(A, Ts),
						todosTienenTransicionesSalientes(EstadosNoFinales, Ts), 
						todosAlcanzables(A, EstadosNoIniciales), 
						hayEstadoFinal(Finales), 
						noHayFinalesRepetidos(Finales),
						noHayTransicionesRep(Ts).
\end{verbatim}
						
Verifica que todos los estados tienen transiciones salientes exceptuando los finales, que pueden
o no tenerlas.
\begin{verbatim}
%todosTienenTransicionesSalientes(+Estados, +Transiciones)
todosTienenTransicionesSalientes(Es, Ts) :- forall(member(E,Es), member((E, _, _), Ts)).
\end{verbatim}

Verifica que todos los estados son alcanzables desde el estado inicial.
\begin{verbatim}
%todosAlcanzables(+Automata, +Es)
todosAlcanzables(A, Es) :- forall(member(E, Es), alcanzable(A,E)). 
\end{verbatim}

Se comprueba que el automata tiene al menos un estado final.
\begin{verbatim}
%hayEstadoFinal(+Fs)
hayEstadoFinal(Fs) :- length(Fs, N), N > 0.
\end{verbatim}

Se chequea que no hay estados finales repetidos.
\begin{verbatim}
%noHayFinalesRepetidos(+Fs)
noHayFinalesRepetidos(Fs) :- sacarDup(Fs, FsSinDup), length(Fs, N), length(FsSinDup, N).
\end{verbatim}

Se fija que no haya transiciones repetidas.
\begin{verbatim}
%noHayTransicionesRep(+Ts)
noHayTransicionesRep(Ts) :- sacarDup(Ts, TsSinDup), length(Ts, N), length(TsSinDup, N).
\end{verbatim}

Ejemplo:


\subsection{Ejercicio 8}

Se utiliza Generate & Test.

\begin{verbatim}
% hayCiclo(+Automata)
hayCiclo(A) :- estados(A, Es), length(Es, M), P is M + 1, member(E, Es),
				between(2, P, N), caminoDeLongitud(A, N, _, _, E, E), !.

\end{verbatim}

Ejemplo:

\subsection{Ejercicio 9}

Se utiliza Generate & Test.
\begin{verbatim}
% reconoce(+Automata, ?Palabra)
reconoce(A, P) :- nonvar(P), length(P,N), inicialDe(A,Si), finalesDe(A,Sfs), 
					member(Sf,Sfs), Nm1 is N+1, caminoDeLongitud(A,Nm1,_,TmpP,Si,Sf), sameList(P,TmpP).

reconoce(A, P) :- var(P), not(hayCiclo(A)), inicialDe(A,Si), finalesDe(A,Sfs), 
					transicionesDe(A,T), length(T,N), Nm1 is N+1, between(1, Nm1, Tam), member(Sf,Sfs), caminoDeLongitud(A,Tam,_,P,Si,Sf).

reconoce(A, P) :- var(P), hayCiclo(A), inicialDe(A,Si), finalesDe(A,Sfs), 
					desde(1, N), member(Sf,Sfs), caminoDeLongitud(A,N,_,P,Si,Sf).

\end{verbatim}

Ejemplo:


\subsection{Ejercicio 10}

Se utiliza Generate & Test.

\begin{verbatim}
% PalabraMásCorta(+Automata, ?Palabra)
palabraMasCorta(A, Palabra) :- nonvar(Palabra), reconoce(A,Palabra),  
								transicionesDe(A,T), length(T,N), Nm1 is N+1, inicialDe(A,Si), finalesDe(A,Sfs), 
								between(1,Nm1,Tam), member(Sf,Sfs), caminoDeLongitud(A,Tam,_,_,Si,Sf), !, Tam2 is Tam -1,
								length(Palabra,TamPal), Tam2 = TamPal.
palabraMasCorta(A, Palabra) :- var(Palabra), transicionesDe(A,T), length(T,N), Nm1 is N+1, inicialDe(A,Si), finalesDe(A,Sfs), 
							between(1,Nm1,Tam), member(Sf,Sfs), caminoDeLongitud(A,Tam,_,_,Si,Sf), !, Len is Tam - 1 ,
							reconoceAcotado(A, Palabra, Len).


\end{verbatim}
							
Se utiliza Generate & Test.
Genera todas las palabras de longitud N de un automata dado.
\begin{verbatim}
% reconoceAcotado(+A, ?P, +N)
reconoceAcotado(A, P, N) :- length(P, N), reconoce(A, P).

\end{verbatim}

Ejemplo:


\end{document}
